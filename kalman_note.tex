\documentclass{article}
\usepackage{xcolor}
\usepackage{tikz}
\usepackage{makeidx}
\makeindex
\usepackage{pgfplots}
\pgfplotsset{compat=1.17}
\begin{document}
\title{Kalman filter study notes}
\author{lj}
\date{February 22,2021}
\newcommand{\updateinfo}[1][\today]{\par\vfill\hfill{\scriptsize\color{gray}Last updated on #1}}

\maketitle
\tableofcontents
\updateinfo{}
\newpage
\newpage

\section{Background knowledge}
\subsection{What's problem Kalman algorithm solve}
Kalman algorithm is an optimism estimate algorithm,it have the limit in the Gaussian linear system.
\subsubsection{Gaussian linear system}
The name have two modifier,Gaussian and linear.
\begin{enumerate}
    \item Gaussian\index{love}
        The Gaussian mean the noisy obey to the statistical distribution.
    \item Linear
        This is mean two property,homogeneity and additivity.
\end{enumerate}
\end{document}
